\section{Code Snippet}

\noindent\textbf{Required packages: listtings, xcolor}

The latex code below is to define the style for ``lstset''. If you want to import a code snippet, you need to write the code as Listing~\ref{list:lst} and refers it by ``$\setminus ref\{list:lst\}$''.

%==================4. COMMANDS FOR CODE SNIPPET==================
\begin{lstlisting}[caption=Style definition for lstset]
\definecolor{pblue}{rgb}{0.13,0.13,1}
\definecolor{pgreen}{rgb}{0,0.5,0}
\definecolor{pred}{rgb}{0.9,0,0}
\definecolor{pgray}{rgb}{0.46,0.45,0.48}
\definecolor{ppurple}{rgb}{1,0.2,1}
\definecolor{pblack}{rgb}{0,0,0}
\lstset{
	basicstyle=\scriptsize\tt,
	tabsize=4,
	showstringspaces=false,
	columns=flexible,
	commentstyle=\color{pgreen},
  	keywordstyle=\color{pblue},
  	stringstyle=\color{ppurple},
	breaklines=true,
	language=Java,
    showspaces=false,
    numbers=left, 
    numbersep=5pt,
    numberstyle=\tiny\color{pblack},
    frame=single
}

\end{lstlisting}

% where to put the line-numbers; possible values are (none, left, right)
% how far the line-numbers are from the code
% the style that is used for the line-numbers
\begin{lstlisting}[label=list:lst,caption=Example for lstset]
\begin{lstlisting}[label=list:lst,caption=Example for lstset]
Your code here
\end{lstlisting }
\end{lstlisting}

\subsection{Putting your code into figure}

\begin{lstlisting}
\begin{figure}
\begin{center}
    \lstinputlisting{code/example.java}         %the location of your code
    \caption{An example of Java program}\label{fig:example}
\end{center}
\end{figure}
\end{lstlisting}

\subsection{Setting width of frame}

\begin{lstlisting}[label=list:lst,caption=Setting width of frame,linewidth=0.8\textwidth]
\begin{lstlisting}[label=list:lst,caption=Setting with of frame,linewidth=0.8\textwidth]
Your code here
\end{lstlisting }
\end{lstlisting}

You can set ``\textsf{xleftmargin}'' and ``\textsf{xrightmargin}'' properties to adjust the position in the page.
%Use the following to cite: Fig.~\ref{fig:example}

\subsection{Frame style of code}

The style of frame can be adjusted by the property ``\textsf{frame}''. The frame is not shown by default or setting the property ``\textsf{frame}'' to ``none''.

\begin{lstlisting}[label=list:lst,caption=Frame style of the code,frame=shadowbox]
\begin{lstlisting}[label=list:lst,caption=Frame style of the code,frame=shadowbox]
Your code here
\end{lstlisting }
\end{lstlisting}