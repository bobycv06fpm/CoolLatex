\documentclass{report}%[sigconf]{acmart}

\usepackage{booktabs} % For formal tables
\usepackage{listings}
\usepackage{xcolor}
\usepackage{enumitem}
\usepackage{tikz}
\usepackage{amsmath,amssymb}

\usetikzlibrary{calc,decorations.pathmorphing,shapes}
\newcounter{sarrow}
\newcommand\xrsquigarrow[1]{%
\stepcounter{sarrow}%
\mathrel{\begin{tikzpicture}[baseline= {( $ (current bounding box.south) + (0,-0.5ex) $ )}]
\node[inner sep=.5ex] (\thesarrow) {$\scriptstyle #1$};
\path[draw,<-,decorate,
  decoration={zigzag,amplitude=0.7pt,segment length=1.2mm,pre=lineto,pre length=4pt}]
    (\thesarrow.south east) -- (\thesarrow.south west);
\end{tikzpicture}}%
}

\begin{document}
\title{SIG Proceedings Paper in LaTeX Format}




%\begin{abstract}
%This paper provides a sample of a \LaTeX\ document which conforms,
%somewhat loosely, to the formatting guidelines for
%ACM SIG Proceedings.\footnote{This is an abstract footnote}
%\end{abstract}

%
% The code below should be generated by the tool at
% http://dl.acm.org/ccs.cfm
% Please copy and paste the code instead of the example below.
%



\maketitle

%==================2. COMMANDS FOR MODIFICATIONS TO THE PAPER==================
\ifdefined \WithChanges
    \newcommand{\todo}[1]{\textbf{\textcolor{blue}{TODO: #1}}}
    %\newcommand{\comment}[1]{\textbf{\textcolor{green}{Comment: #1}}}
    \newcommand{\alarm}[1]{\textbf{\textcolor{red}{Alarm: #1}}}
    \newcommand{\deprecate}[1]{\textbf{\textcolor{red}{\{Deprecated: #1\}}}}
    \newcommand{\question}[1]{\textbf{\textcolor{blue}{Question: #1}}}
    \newcommand{\remove}[1]{}
\else
    \newcommand{\todo}[1]{\textbf{\textcolor{blue}{TODO: #1}}}
    %\newcommand{\comment}[1]{\textbf{\textcolor{green}{Comment: #1}}}
    \newcommand{\alarm}[1]{\textbf{\textcolor{red}{Alarm: #1}}}
    \newcommand{\deprecate}[1]{\textbf{\textcolor{red}{\{Deprecated: #1\}}}}
    \newcommand{\question}[1]{}
    \newcommand{\remove}[1]{}
\fi

%==================3. COMMANDS FOR DEFINITIONS==================
\newtheorem{mydef}{Definition}%[section]
%Example for definitions:
%   \begin{mydef}\label{thm:example}
%       This is a definition.
%   \end{mydef}
%Use the following to cite: Definition~\ref{thm:example}

%==================4. COMMANDS FOR CODE SNIPPET==================
\definecolor{pblue}{rgb}{0.13,0.13,1}
\definecolor{pgreen}{rgb}{0,0.5,0}
\definecolor{pred}{rgb}{0.9,0,0}
\definecolor{pgray}{rgb}{0.46,0.45,0.48}
\definecolor{ppurple}{rgb}{1,0.2,1}
\definecolor{pblack}{rgb}{0,0,0}
\lstset{
	basicstyle=\scriptsize\tt,
	tabsize=4,
	showstringspaces=false,
	columns=flexible,
	commentstyle=\color{pgreen},
  	keywordstyle=\color{pblue},
  	stringstyle=\color{ppurple},
	breaklines=true,
	language=Java,
    showspaces=false,
    numbers=left,                    % where to put the line-numbers; possible values are (none, left, right)
    numbersep=5pt,                   % how far the line-numbers are from the code
    numberstyle=\tiny\color{pblack}, % the style that is used for the line-numbers
    frame=single
}
%Example for code snippet:
%\begin{figure}
%\begin{center}
%    \lstinputlisting{code/example.java}
%    \caption{An example of Java program}\label{fig:example}
%\end{center}
%\end{figure}
%Use the following to cite: Fig.~\ref{fig:example}

%==================4. COMMANDS TO REDUCE THE SPACE==================
\ifdefined \ReduceSpace
    \addtolength{\parskip}{-1mm}
    \addtolength{\floatsep}{-6mm}
    \addtolength{\textfloatsep}{-6mm}
    \addtolength{\abovecaptionskip}{-0.5mm}
    \addtolength{\belowcaptionskip}{-0.5mm}
\fi

%==================5. COMMANDS FOR SQUIGGLE ARROW==================
%Source: http://tex.stackexchange.com/questions/60216/how-to-create-a-squiggle-arrow-with-some-text-on-it-in-tikz
%\newcounter{sarrow}
%\newcommand\xrsquigarrow[1]{%
%\stepcounter{sarrow}%
%\mathrel{\begin{tikzpicture}[baseline= {( $ (current bounding box.south) + (0,-0.5ex) $ )}]
%\node[inner sep=.5ex] (\thesarrow) {$\scriptstyle #1$};
%\path[draw,<-,decorate,
%  decoration={zigzag,amplitude=0.7pt,segment length=1.2mm,pre=lineto,pre length=4pt}]
%    (\thesarrow.south east) -- (\thesarrow.south west);
%\end{tikzpicture}}%
%}
%Example for squiggle arrow
%   $s_1 \xrsquigarrow{e} s_2$

%==================6. COMMANDS FOR SPACE BEFORE ITEMIZE==================
\setitemize[0]{leftmargin=10pt}

%==================7. COMMANDS FOR CODE IN PAPER==================
\newcommand{\fonttt}[1]{\begin{ttfamily}#1\end{ttfamily}}
%\textsf{\small ClassName}
%\textsc{TOOL}
\newcommand{\ourtool}{\textsc{EvoMal}}

%=======================8. Change the width of line in table ====================
%\Xhline{2\arrayrulewidth}
\def\checkmark{\tikz\fill[scale=0.4](0,.35) -- (.25,0) -- (1,.7) -- (.25,.15) -- cycle;}

%=======================9. Scale the width of table if beyond the width ====================
%\usepackage{adjustbox}
%\begin{table}
%\begin{center}
%\caption{Sample}\label{tbl:sample}
%\begin{adjustbox}{width=0.48\textwidth}
%\begin{tabular}{|c|c|}
%\hline
%a & b  \\ \hline
%1 & 2  \\ \hline
%\end{tabular}
%\end{adjustbox}
%\end{center}
%\end{table}

%========================10. the pie chart===========================
\newcommand{\pie}[1]{%
\begin{tikzpicture}
 \draw (0,0) circle (0.8ex);
 \ifthenelse{#1 > 0}{\fill (0.8ex,0) arc (0:#1:0.8ex) -- (0,0) -- cycle;}{}
\end{tikzpicture}%
}
%Example: \pie{} 

%=================11. breakline for a long text=======================
\newcommand{\hash}[1]{\texttt{\zz#1\zz}}

\def\zz#1{%
 \ifx\zz#1\else
   #1\linebreak[1]\expandafter\zz
 \fi}

\section{Arrow}

\subsection{Squiggle arrow}

\noindent\textbf{Required packages: tikz, amsmath, amssymb}
%Source: http://tex.stackexchange.com/questions/60216/how-to-create-a-squiggle-arrow-with-some-text-on-it-in-tikz

The squiggle arrow is drawn by \textsf{tikz}~\cite{squigglearrow}, and the code is as follows:

\begin{lstlisting}
\documentclass{...}
\usepackage{tikz}
\usepackage{amsmath,amssymb}

\usetikzlibrary{calc,decorations.pathmorphing,shapes}
\newcounter{sarrow}
\newcommand\xrsquigarrow[1]{%
\stepcounter{sarrow}%
\mathrel{\begin{tikzpicture}[baseline= {( $ (current bounding box.south) + (0,-0.5ex) $ )}]
\node[inner sep=.5ex] (\thesarrow) {$\scriptstyle #1$};
\path[draw,<-,decorate,
  decoration={zigzag,amplitude=0.7pt,segment length=1.2mm,pre=lineto,pre length=4pt}]
    (\thesarrow.south east) -- (\thesarrow.south west);
\end{tikzpicture}}%
}

\begin{document}
\[ s_1 \xrsquigarrow{e} s_2 \]
\end{document}
\end{lstlisting}

With the above code, you can draw a squiggle arrow as follows:

\[ s_1 \xrsquigarrow{e} s_2 \]

With the following code, you can draw more styles of arrows as follows:

\begin{lstlisting}
\[
A\xrightarrow{f} B\quad A\rightsquigarrow B\quad A\xrsquigarrow{f}B\quad A\xrsquigarrow{(f\circ g)\circ h}B
\]
\end{lstlisting}

\[
A\xrightarrow{f} B\quad A\rightsquigarrow B\quad A\xrsquigarrow{f}B\quad A\xrsquigarrow{(f\circ g)\circ h}B
\]

\section{Code Snippet}

\noindent\textbf{Required packages: listtings, xcolor}

The latex code below is to define the style for ``lstset''. If you want to import a code snippet, you need to write the code as Listing~\ref{list:lst} and refers it by ``$\setminus ref\{list:lst\}$''.

%==================4. COMMANDS FOR CODE SNIPPET==================
\begin{lstlisting}[caption=Style definition for lstset]
\definecolor{pblue}{rgb}{0.13,0.13,1}
\definecolor{pgreen}{rgb}{0,0.5,0}
\definecolor{pred}{rgb}{0.9,0,0}
\definecolor{pgray}{rgb}{0.46,0.45,0.48}
\definecolor{ppurple}{rgb}{1,0.2,1}
\definecolor{pblack}{rgb}{0,0,0}
\lstset{
	basicstyle=\scriptsize\tt,
	tabsize=4,
	showstringspaces=false,
	columns=flexible,
	commentstyle=\color{pgreen},
  	keywordstyle=\color{pblue},
  	stringstyle=\color{ppurple},
	breaklines=true,
	language=Java,
    showspaces=false,
    numbers=left, 
    numbersep=5pt,
    numberstyle=\tiny\color{pblack},
    frame=single
}

\end{lstlisting}

% where to put the line-numbers; possible values are (none, left, right)
% how far the line-numbers are from the code
% the style that is used for the line-numbers
\begin{lstlisting}[label=list:lst,caption=Example for lstset]
\begin{lstlisting}[label=list:lst,caption=Example for lstset]
Your code here
\end{lstlisting }
\end{lstlisting}

\subsection{Putting your code into figure}

\begin{lstlisting}
\begin{figure}
\begin{center}
    \lstinputlisting{code/example.java}         %the location of your code
    \caption{An example of Java program}\label{fig:example}
\end{center}
\end{figure}
\end{lstlisting}

\subsection{Setting width of frame}

\begin{lstlisting}[label=list:lst,caption=Setting width of frame,linewidth=0.8\textwidth]
\begin{lstlisting}[label=list:lst,caption=Setting with of frame,linewidth=0.8\textwidth]
Your code here
\end{lstlisting }
\end{lstlisting}

You can set ``\textsf{xleftmargin}'' and ``\textsf{xrightmargin}'' properties to adjust the position in the page.
%Use the following to cite: Fig.~\ref{fig:example}

\subsection{Frame style of code}

The style of frame can be adjusted by the property ``\textsf{frame}''. The frame is not shown by default or setting the property ``\textsf{frame}'' to ``none''.

\begin{lstlisting}[label=list:lst,caption=Frame style of the code,frame=shadowbox]
\begin{lstlisting}[label=list:lst,caption=Frame style of the code,frame=shadowbox]
Your code here
\end{lstlisting }
\end{lstlisting}

%\bibliographystyle{ACM-Reference-Format}
\bibliography{sigproc}

\end{document}
