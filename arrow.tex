\section{Arrow}

\subsection{Squiggle arrow}

\noindent\textbf{Required packages: tikz, amsmath, amssymb}
%Source: http://tex.stackexchange.com/questions/60216/how-to-create-a-squiggle-arrow-with-some-text-on-it-in-tikz

The squiggle arrow is drawn by \textsf{tikz}~\cite{squigglearrow}, and the code is as follows:

\begin{lstlisting}
\documentclass{...}
\usepackage{tikz}
\usepackage{amsmath,amssymb}

\usetikzlibrary{calc,decorations.pathmorphing,shapes}
\newcounter{sarrow}
\newcommand\xrsquigarrow[1]{%
\stepcounter{sarrow}%
\mathrel{\begin{tikzpicture}[baseline= {( $ (current bounding box.south) + (0,-0.5ex) $ )}]
\node[inner sep=.5ex] (\thesarrow) {$\scriptstyle #1$};
\path[draw,<-,decorate,
  decoration={zigzag,amplitude=0.7pt,segment length=1.2mm,pre=lineto,pre length=4pt}]
    (\thesarrow.south east) -- (\thesarrow.south west);
\end{tikzpicture}}%
}

\begin{document}
\[ s_1 \xrsquigarrow{e} s_2 \]
\end{document}
\end{lstlisting}

With the above code, you can draw a squiggle arrow as follows:

\[ s_1 \xrsquigarrow{e} s_2 \]

With the following code, you can draw more styles of arrows as follows:

\begin{lstlisting}
\[
A\xrightarrow{f} B\quad A\rightsquigarrow B\quad A\xrsquigarrow{f}B\quad A\xrsquigarrow{(f\circ g)\circ h}B
\]
\end{lstlisting}

\[
A\xrightarrow{f} B\quad A\rightsquigarrow B\quad A\xrsquigarrow{f}B\quad A\xrsquigarrow{(f\circ g)\circ h}B
\]
